\section{Related Work}
Little work can be found in the field of evolutionary algorithms being applied in the context of plants, such as the generation and optimization of L-systems. In the context of games, work has been done to utilize evolutionary computation to make games more immersive, and in the creation of buildings, environmental impacted flora, and dynamic weapons\cite{wonka}\cite{shaker}\cite{visualmodels}, however, there is a lack of such a method for the generation of realistic flora \cite{evolving}. 

The first of the few works that has been conducted on evolutionary algorithms for plants was done by Niklas who used three parameters, along with a constrained evolutionary technique implementing the nearest neighbor heuristic \cite{d9683286-f374-3c0d-ae3c-3924235e199a}. However, another paper by Jacob restricts the production to a set number of branches, but proves that the use of genetic algorithms for the generation of L-systems is a viable method in the production of 3-dimensional plants \cite{evolprograms}. 

Other works used 2-dimensional DOL-Systems (Deterministic and Context free) that are the simplest L-Systems using an advanced fitness function in order mimic the natural growth of plants \cite{ochoa1998genetic}, and implement an interactive process by selecting the best individuals as the selection process. In 1998, Mock worked on the evolution of Wildwood's plants, which are based on L-systems, by generating a random initial population, incorporating both computational evolution and hand-bred human based phenotypes in his fitness function, and selecting the best individuals as the selection process \cite{wildwood}. This approach helped beginners to test and play around with plant-like structures and their evolutions based on certain features as per their requirements.  

% but all in all, the conclusion is similar that there is need for more research in the field of evolutionary algorithms for the generation of realistic plants. 