\section{Introduction}
The field of computational intelligence has seen significant advancements in recent years, with applications ranging from data analysis to artificial intelligence, and even various engineering feilds \cite{ethan}\cite{slowik2020evolutionary}. Computational techniques inspired from nature can play a pivotal role in various fields such as animations, computer graphics, games, and virtual reality. Several techniques effectively model not only real world structures, but also living beings and real life lower organisms \cite{dawkins}\cite{oppenheimer}\cite{sims}.

L-systems, or Lindenmayer Systems, based on the work of Aristid Lindenmayer \cite{LINDENMAYER1968280} provide mathematical models for the development of growth of various complex organisms such as fundi, plants, and other natural systems. First defined as linear arrays of finite automata, later gained momentum and developed into a branch of formal langauge theory \cite{rozenberg1980mathematical}, followed by their applications for modeling plants and plant-like structures using rule-based techniques\cite{radomir}.

Our research aims at using Evolutionary Algorithms to simulate the evolution of artificial 2D plant morphologies. Based on various parameters, those structures can be evolved, each time on a random rule, to provide the optimally stable structure for a given set of parameters. The main goal of this research is to optimize the structure of the plant by evolving the rules of the L-systems using Evolutionary Algorithms. 