\section{Evolutionary Algorithm}
The evolutionary algorithm (EA) [1] stands as one of the oldest and most recognized optimization techniques, drawing inspiration from natural processes. In an EA, the exploration of solution space mirrors the dynamics of natural environments, incorporating principles from Darwinian evolution\cite{slowik2020evolutionary}. Within EA, a population of individuals exists; each individual, referred to as a chromosome, embodies a potential solution to the problem at hand. The problem's definition stems from the fitnes function. Based on the degree of alignment with the fitness function, each individual is assigned a quality measure, known as fitness. Fitness serves as a primary criterion for evaluation. Individuals with higher fitness values stand a better chance of being selected for the next generation of the population. EAs encompass three key operators: Survivor Selection (where a new population is formed based on the fitness values of individuals from the previous generation), crossover (which typically involves exchanging parts of individuals between selected pairs), and mutation (where specific gene values are randomly altered). The parameters of the evolutionary algorithm (EA) require careful fine-tuning to achieve a balance between exploration and exploitation within the search space, facilitating convergence. In our application of the EA to optimize plant structures generated by L-systems, we have focused on ensuring both exploration and exploitation by meticulously adjusting parameters and experimenting with various selection schemes.
\subsection{Population}

In an Evolutionary Algorithm, the population represents a collection of candidate solutions for the optimization problem. For our problem, a chromosome consists of rules of the L-System. Thus, the population comprises an array of L-system rules, where each string corresponds to a chromosome. These chromosomes translate into paths taken by the turtle, represented as an array of Cartesian coordinates $(x, y, \theta)$.

The length of each chromosome ranges from 5 to 30 characters initially. However, as the algorithm proceeds and substitutions occur, the chromosome length may increase with each generation. Unlike in traditional Evolutionary Algorithms, we cannot impose strict constraints on the chromosome length during evolution, as doing so could impede the functionality of the L-system.


\subsection{Fitness Functions}
In order to evolve the structures to make them look like real trees, we looked at past research into botany and what factors or genes really contribute to a well evolved tree structure. We realised that natural selection operates on phenotypic effects rather than genes directly. Human perception excels at selecting phenotypic patterns, but creating computer programs to do so is complex.

To achieve a higher degree of automation, it was necessary to design an effective fitness function. Formulating hypotheses regarding the factors influencing plant evolution was crucial for guiding simulated evolution toward plant-like structures. We came up with 5 behaviors demontrated by plants: Vertical Tropism, Bilateral Symmetry, Stability, structural stability, Photon Harvesting Ability and Branching Proliferation\cite{d9683286-f374-3c0d-ae3c-3924235e199a}\cite{ochoa1998genetic}.

Consequently, we have based our fitness functions on that. These components are quantified using simple algorithmic techniques based on geometric information derived from the L-system's interpretation in a 2D Cartesian coordinate system.

\begin{itemize}
\item \textbf{Vertical Tropism}
Vertical Tropism assesses the upward growth potential of plants. Taller plants are often more adept at accessing sunlight for photosynthesis and spreading seeds, thus promoting their growth and propagation\cite{d9683286-f374-3c0d-ae3c-3924235e199a}.

\item \textbf{Bilateral Symmetry}
Structural Symmetry assesses the balance and proportionality of plant structures. The 'weight' balance of the structure is estimated by summing the absolute values of vertices' x-coordinates on both sides of the vertical axis. Higher fitness is assigned to structures whose left-to-right ratio is closer to one, indicating better balance and symmetry. This ensures that the plant exhibits a well-balanced branching pattern on both sides of the trunk, enhancing stability and aesthetic appeal.

\item \textbf{Stability}
Stability measures the structural integrity of plants by analyzing the number of branches emerging from branching points. Plants with fewer branches emanating from each point are considered more stable, as excessive branching can lead to structural weakness.

\item \textbf{Photon Harvesting Ability}
Photon Harvesting Ability quantifies the plant's efficiency in capturing sunlight for photosynthesis. This is assessed by determining the maximum absolute value of the x-coordinate along the plant structure. Greater spread or surface area of branches results in higher photon harvesting ability, as it allows the plant to capture more sunlight for energy production.

\item \textbf{Branching Proliferation}
Branching Proliferation examines the density and abundance of branches in plant structures. Plants with a higher proportion of branching points and multiple branches demonstrate enhanced capacity for resource acquisition and reproduction.
\end{itemize}

\subsubsection*{Overall Fitness Formula}
In order to calculate the combined fitness we used the following formula:
\\

\begin{large}
    $\frac{(vt \times w_{vt}) + (ss \times w_{bs}) + (s \times w_s) + (pha \times w_{pha}) + (bp \times w_{bp})}{w_{vt} + w_{bs} + w_s + w_{pha} + w_{bp}}$
\end{large}
\\
\\ 


Where:\\
\begin{itemize}
    \item $w_{vt}$: Weight representing the importance of Vertical Tropism
    \item $w_{bs}$: Weight representing the importance of Bilateral Symmetry
    \item $w_s$: Weight representing the importance of Stability
    \item $w_{pha}$: Weight representing the importance of Photon Harvesting Ability
    \item $w_{bp}$: Weight representing the importance of Branching Proliferation
    \item $vt$: Fitness value for Vertical Tropism
    \item $ss$: Fitness value for Structural Symmetry
    \item $s$: Fitness value for Stability
    \item $pha$: Fitness value for Photon Harvesting Ability
    \item $bp$: Fitness value for Branching Proliferation
\end{itemize}



\subsection{Crossover}
In evolutionary algorithms, crossover is pivotal for creating offspring with diverse genetic traits, mimicking natural genetic recombination. Our implementation utilizes a two-point crossover approach, where segments between randomly chosen points in parent chromosomes are exchanged. This process fosters the generation of offspring that inherit a combination of parental characteristics, potentially leading to novel and promising solutions. Valid offspring are iteratively produced until two suitable candidates emerge, contributing to the population's genetic diversity and the algorithm's exploration of solution space. By integrating crossover, our algorithm enhances the search for optimal solutions by leveraging the genetic information encoded in the parent chromosomes.\\

Consider two parent chromosomes:

\textbf{Parent 1:} 'F+F-[F-]' \\
\textbf{Parent 2:} 'F-F+[F+]'

Crossover points are randomly selected within each parent chromosome, and genetic material is swapped between them. 

Offspring 1 inherits segments from Parent 1 and Parent 2, resulting in 'F+F-[FF-]'. Offspring 2 inherits segments from Parent 1 and Parent 2, resulting in 'F[F+]F+'.

This 2-point crossover introduces genetic diversity, potentially leading to the discovery of novel solutions.

\subsection{Mutation}

In our mutation operation, we employed a block mutation approach to introduce variability within the offspring chromosomes. This method involves selecting a contiguous block of genetic material within the chromosome and replacing it with a new sequence of characters. The selected block represents a segment of the chromosome that undergoes alteration, potentially leading to the exploration of new genetic configurations.

For example, consider the chromosome 'F+F-[-FF+]'. During mutation, we may choose to replace the block '[-FF+]' with a new sequence, such as 'FFF'. As a result, the mutated chromosome becomes 'F+F-FFF'. This process allows for the introduction of diversity within the population, facilitating the exploration of alternative solutions and potentially improving the evolutionary process.

\subsection{Parent Selection Schemes}

Parent selection is a crucial step in evolutionary algorithms where individuals from the population are chosen to become parents for the next generation. Here, we explored different parent selection schemes utilized in our system:

\begin{itemize}
    \item \textbf{Truncation Selection}: Truncation selection involves selecting the top individuals from the population based on their fitness scores. In our implementation, we select parents from the top 10\% of the population. This method aims to maintain genetic diversity by favoring individuals with higher fitness, potentially leading to faster convergence towards optimal solutions.

    \item \textbf{Fitness Proportionate Selection}: Fitness proportionate selection, also known as roulette wheel selection, selects parents with probabilities proportional to their fitness values. Individuals with higher fitness have a higher chance of being chosen as parents. This method aims to strike a balance between exploitation of high-fitness individuals and exploration of the search space.

    \item \textbf{Rank-Based Selection}: Rank-based selection assigns probabilities to individuals based on their ranks in the population. Higher-ranked individuals have higher probabilities of being selected as parents. This approach is less sensitive to outliers in fitness values and helps maintain diversity by ensuring that lower-ranked individuals still have a chance of being selected.

    \item \textbf{Tournament Selection}: Tournament selection involves randomly selecting a subset of individuals from the population and choosing the fittest individual from that subset as a parent. This process is repeated to select multiple parents. Tournament selection is robust and less prone to premature convergence, making it suitable for maintaining diversity in the population.

    \item \textbf{Random Selection}: Random selection randomly chooses individuals from the population to be parents. In this scheme, each individual has an equal chance of being selected. While simple, this method may not necessarily prioritize fitter individuals, potentially leading to slower convergence or premature convergence to suboptimal solutions.
    
    
\end{itemize}

\subsection{Survivor Selection Schemes}

These are the survivor schemes used to determine which individuals from the current population will survive to the next generation:

\begin{itemize}
    \item \textbf{Binary Tournament}: In our implementation, binary tournament selection was utilized, where two individuals are randomly selected from the population, and the fitter individual is chosen as a survivor. This process is repeated until the desired number of survivors is reached. Binary tournament selection provides a simple and efficient way to balance exploration and exploitation.
    
    \item \textbf{Truncation Selection}: Truncation selection involves keeping the top individuals from the population based on their fitness scores. In our implementation, we retained all individuals, ensuring that the fittest individuals continued to contribute to the population in the next generation. Truncation selection proved effective for maintaining high-quality solutions over multiple generations.
    
    \item \textbf{Fitness Proportional Selection}: We used fitness proportional selection, also known as roulette wheel selection, to select survivors with probabilities proportional to their fitness values. Individuals with higher fitness had a higher chance of being chosen as survivors. This method maintained diversity in the population while favoring individuals with higher fitness.
    
    \item \textbf{Rank-Based Selection}: Rank-based selection assigns ranks to individuals based on their fitness values and selects survivors based on their ranks. Higher-ranked individuals had a higher chance of being chosen as survivors. Rank-based selection was robust and less sensitive to outliers in fitness values, ensuring a fair representation of individuals in the next generation.
\end{itemize}



