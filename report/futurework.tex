\section{Future Work}
In our implementation, a chromosome represents rule, which is dynamically updated by substitution in the genetic algorithm a certain number of itmes. This results in the updated string being increased by a significant factor in a very short amount of time. Thus, parent selection schemes that need to calculate the fitness of the entire, or a significant portion of the population take a considerable amount of time. In particular, the truncation scheme for parent selection shows significant performance degradation for very small number of generations of about 20 due to the increased string length. A similar degradation is seen in the fitness proportionate selection scheme, which also requires the fitness of the entire population to be calculated. Due to this, we were unable to run the truncation scheme for more than 20 generations, compared to 300 generations for other selection schemes. Access to more computational resources instead of our laptops, such as a high-performance computer or computing cluster would allow us to run the truncation scheme for a longer period of time, and potentially observe better results. 

Further, most of our testing for the new rules was done for the tree-based structure. The fitness functions were also designed to optimize the tree-based structure. The same is being used for the Sierpenski triangle and the Dragon Curve. Future work could include specifically testing the new rules for these structures, and designing new and better rules and fitness functions tailored to these structures based on the patterns that they exhibit. This would undoubtedly give even better results for these structures.

Lastly, the colour of the structure can also be taken into account when designing the fitness function. This can be used as a parameter in the fitness function to determine the fitness of the structure. Taking the colour as a parameter and evolving it would allow for the generation of more aesthetic, visually appealing, beautiful, and even realistic structures. This could also result in taking a step towards the generation of realistic flora.